%Example of use of oxmathproblems latex class for problem sheets
\documentclass{oxmathproblems}

%(un)comment this line to enable/disable output of any solutions in the file
%\printanswers

%define the page header/title info
\course{Algorithm Design and Analysis}
\sheettitle{Assignment 6 \\ Deadline: Jun 20, 2024} %can leave out if no title per sheet


\begin{document}
Choose \textbf{two} questions to solve.

\begin{questions}
\miquestion[50] 
Given an undirected graph $G=(V,E)$ and an integer $k$, decide if $G$ has a spanning tree with maximum degree at most $k$. Prove that this problem is NP-complete.

Answer:

To show that the problem of deciding if a given undirected graph \( G = (V, E) \) has a spanning tree with maximum degree at most \( k \) is NP-complete, we need to do two things:

1. Prove that the problem is in NP.

2. Prove that the problem is NP-hard by reducing a known NP-complete problem to it.

\textbf{1. The Problem is in NP}

To prove that the problem is in NP, we need to show that given a certificate (in this case, a spanning tree of \( G \) with maximum degree at most \( k \)), we can verify in polynomial time that the certificate is valid.

\textbf{Certificate:} A spanning tree \( T \) of \( G \) where the maximum degree of any vertex in \( T \) is at most \( k \).

\textbf{Verification:} We can verify this by:

  1. Checking that \( T \) is a subgraph of \( G \).
  
  2. Checking that \( T \) includes all vertices of \( G \) and that \( T \) is connected.
  
  3. Checking that \( T \) has exactly \( |V| - 1 \) edges (which is the property of a tree).
  
  4. Checking that the degree of every vertex in \( T \) is at most \( k \).

Each of these checks can be performed in polynomial time, so the problem is in NP.

\textbf{2. The Problem is NP-Hard}

To prove that the problem is NP-hard, we need to show a polynomial-time reduction from a known NP-complete problem to our problem. We will reduce from the \textbf{Hamiltonian Path problem}, which is known to be NP-complete.

\textbf{Reduction}

1. \textbf{Input:} A graph \( G' = (V', E') \) for the Hamiltonian Path problem.

2. \textbf{Transformation:}

   - Construct a new graph \( G \) from \( G' \) by adding a new vertex \( v_0 \) and connecting \( v_0 \) to every vertex in \( V' \). Let \( G = (V, E) \), where \( V = V' \cup \{v_0\} \) and \( E = E' \cup \{ (v_0, v) \mid v \in V' \} \).
   
   - Set \( k = 2 \).
\newpage
3. \textbf{Correctness:}

   Suppose \( G' \) has a Hamiltonian Path. Then there exists a path \( P \) in \( G' \) visiting each vertex exactly once. We can construct a spanning tree \( T \) in \( G \) as follows:
   
     - Include the edges of \( P \) in \( T \).
     
     - Add the edge from \( v_0 \) to one end of \( P \).

     In this tree \( T \), the degree of \( v_0 \) is 1, and the degree of every other vertex is at most 2 (since they were part of a Hamiltonian Path and each internal vertex in the path has degree 2, while the endpoints have degree 1). Hence, the maximum degree in \( T \) is 2.

   Conversely, suppose \( G \) has a spanning tree \( T \) with maximum degree at most 2. Since \( T \) is a spanning tree, \( T \) must include \( v_0 \). The degree of \( v_0 \) must be 1 (since the maximum degree is 2 and \( v_0 \) connects to all other vertices). Removing \( v_0 \) and its incident edge from \( T \) leaves a path in \( G' \) that visits all vertices exactly once (a Hamiltonian Path).

Since the transformation can be performed in polynomial time and the Hamiltonian Path problem is NP-complete, the problem of determining if \( G \) has a spanning tree with maximum degree at most \( k \) is NP-hard.

\miquestion[50] 
An \emph{0-1 integer linear program} is similar to a linear program, except that each variable $x_i$ is required to be either 0 or 1.
\begin{align*}
   \text{maximize} & \quad c_1x_1+\cdots+c_nx_n\\
   \text{Subject to} & \quad a_{11}x_1+\cdots + a_{1n}x_n\leq b_1\\
   &\quad a_{21}x_1+\cdots+a_{2n}x_n\leq b_2\\
   &\qquad\vdots\\
   &\quad a_{m1}x_1+\cdots+a_{mn}x_n\leq b_m\\
   &\quad x_1,\ldots,x_n\in\{0,1\}
\end{align*}
We have seen in the class that a linear program can be solved in polynomial time. However, we will see in this question that this is unlikely for 0-1 integer linear programs.
\begin{parts}
\part Prove that, for a given input $k$, deciding if there is a feasible solution $(x_1,\ldots,x_n)$ such that $c_1x_1+\cdots+c_nx_n\geq k$ is NP-complete.
\part Prove that it is NP-complete to even decide if there is a feasible solution.
\end{parts}
\newpage
Answer:


\textbf{Part (a):} 

\textbf{1. The Problem is in NP:}

   Certificate: A vector \( x = (x_1, x_2, \ldots, x_n) \) where each \( x_i \in \{0, 1\} \).
   
   Verification: Given \( x \):
   
     - Check if \( a_{11} x_1 + a_{12} x_2 + \cdots + a_{1n} x_n \leq b_1 \) for each constraint.
     
     - Check if \( c_1 x_1 + c_2 x_2 + \cdots + c_n x_n \geq k \).

   These checks can be performed in polynomial time, so the problem is in NP.

\textbf{2. The Problem is NP-hard:}

   We will reduce from the \textbf{Subset Sum} problem, which is known to be NP-complete.

\textbf{3. Reduction from Subset Sum to 0-1 integer linear program:}

   Given an instance of the Subset Sum problem, construct the following 0-1 ILP:
   
     Objective function: Maximize \( \sum_{i=1}^n a_i x_i \).
     
     Constraint: \( \sum_{i=1}^n a_i x_i \leq S \).
     
     Set \( k = S \).
   
   The resulting 0-1 ILP is:
   \[
   \text{maximize} \quad \sum_{i=1}^n a_i x_i
   \]
   \[
   \text{subject to} \quad \sum_{i=1}^n a_i x_i \leq S
   \]
   \[
   x_i \in \{0, 1\} \quad \forall i
   \]

   If there is a subset of \( \{a_1, a_2, \ldots, a_n\} \) that sums to \( S \), then there is a feasible solution to the 0-1 ILP where the objective value is at least \( S \). Conversely, if the 0-1 ILP has a feasible solution with the objective value at least \( S \), then there is a subset of the given integers that sums to \( S \).
   This shows the equivalence of the two problems and thus proves that the 0-1 ILP problem is NP-hard.

Since the problem is in NP and NP-hard, it is NP-complete.

\textbf{Part (b):}

\textbf{1. The Problem is in NP:}

   Certificate: A vector \( x = (x_1, x_2, \ldots, x_n) \) where each \( x_i \in \{0, 1\} \).
   
   Verification: Given \( x \):
   
     - Check if \( a_{11} x_1 + a_{12} x_2 + \cdots + a_{1n} x_n \leq b_1 \) for each constraint.

   These checks can be performed in polynomial time, so the problem is in NP.

\textbf{The Problem is NP-hard:}

   We will reduce from the \textbf{3-SAT} problem, which is known to be NP-complete.

\textbf{3. Reduction from 3-SAT to 0-1 ILP:}

   Given an instance of 3-SAT, construct the following 0-1 ILP:
   
     For each variable \( x_i \) in the 3-SAT instance, introduce a corresponding variable \( x_i \in \{0, 1\} \) in the ILP.For each clause in  3-SAT instance, construct a constraint in  ILP such that the constraint is satisfied if and only if at least one of the literals in the clause is true.

   For example, consider a clause \( (x_1 \lor \neg x_2 \lor x_3) \). The corresponding ILP constraint could be:
   \[
   x_1 + (1 - x_2) + x_3 \geq 1
   \]

   This constraint ensures that at least one of \( x_1 \), \( \neg x_2 \), or \( x_3 \) is true (equivalent to \( x_1 = 1 \) or \( x_2 = 0 \) or \( x_3 = 1 \)).

   Repeat this for all clauses in the 3-SAT instance to form the complete ILP.
   The resulting ILP will have a feasible solution if and only if the original 3-SAT instance is satisfiable.

Since 3-SAT is NP-complete and we have shown a polynomial-time reduction to the feasibility problem of a 0-1 ILP, the feasibility problem of a 0-1 ILP is NP-hard.

Since the problem is in NP and NP-hard, it is NP-complete.

\miquestion[50]
Suppose we want to allocate $n$ items $S=\{1,\ldots,n\}$ to two agents.
The two agents may have different values for each item.
Let $u_1,u_2,\ldots,u_n$ be agent $1$'s values for those $n$ items, and $v_1,v_2,\ldots,v_n$ be agent $2$'s values for those $n$ items.
An allocation is a partition $(A,B)$ for $S$, where $A$ is the set of items allocated to agent $1$ and $B$ is the set of items allocated to agent $2$.
An allocation $(A,B)$ is \emph{envy-free} if, based on each agent's valuation, (s)he believes the set (s)he receives is (weakly) more valuable than the set received by the other agent.
Formally, $(A,B)$ is envy-free if
$$\sum_{i\in A}u_i\geq\sum_{j\in B}u_j\qquad\text{agent 1 thinks }A\text{ is more valuable}$$
and
$$\sum_{i\in B}v_i\geq\sum_{j\in A}v_j\qquad\text{agent 2 thinks }B\text{ is more valuable}.$$
Prove that deciding if an envy-free allocation exists is NP-complete.

\miquestion[50]
In the \emph{$k$-means} problem, you are given a set of data points $D=\{\mathbf{x}_1,\ldots,\mathbf{x}_n\in\mathbb{R}^d\}$ and a positive integer $k$ as inputs, and you need to output $k$ ``centers'' $\mathbf{c}_1,\ldots,\mathbf{c}_k\in\mathbb{R}^d$ and a $k$-partition $(C_1,\ldots,C_k)$ of the data points $D$ such that the data points in $C_i$ is assigned to the center $\mathbf{c}_i$.
The objective is to minimize the following value
$$\sum_{i=1}^k\sum_{\mathbf{x}\in C_i}\|\mathbf{x}-\mathbf{c}_i\|^2$$
which is the sum of the squared distances from the data points to their assigned centers.

Prove that the following problem is NP-complete:
given a $k$-means instance $(D,k)$ and a non-negative value $\theta$, decide if there exists a solution $((\mathbf{c}_1,\ldots,\mathbf{c}_k),(C_1,\ldots,C_k))$ that makes the objective value at most $\theta$.

  
  
\miquestion
How long does it take you to finish the assignment (including thinking and discussion)?
Give a score (1,2,3,4,5) to the difficulty.
Do you have any collaborators?
Please write down their names here.
 
To prepare for the final exams, I choose two questions to answer.The NP problem contains so much fun that I will dive into more learning in the near future. Thanks very much for my teacher and teaching assistants' devotion to the class and reviewing my work.Best wishes!


\end{questions}


\end{document}
